% !TeX spellcheck = en_US
%\section{Singular Homology Theory}
Here we will introduce the singular homology theory of arbitrary topological space. The ultimate goal is the \person{Mayer}-\person{Vietoris} sequence and a few applications like \person{Brouwer}-Theorem and nonretractibility of a disk onto its boundary.\\
So lets start with a few definitions
\begin{*definition}
	We are in $\Rn$ and $x,y \in \Rn$ are points. Lets define a few things
	\begin{enumerate}
		\item (line)-\begriff{segment} from $x$ to $y$ to be $\set{(1-t)x + ty \mid t \in [0,1]}$
		\item (a subset) $C \subseteq \Rn$ is \begriff{convex} 
		\item \begriff{convex hull}
		\item \begriff{$p$-simplex} $s$
	\end{enumerate}
\end{*definition}
\todo[inline]{Okay thats worth a proof!}
\begin{*remark}
	\scyllanote{
	\begin{itemize}
		\item Thats crucial here, but need to show that $\bigcap$ of all convex sets in $\Rn$ which contain $A$ are again convex (We can write this as $\CC$ the family of convex sets of $\Rn$). (in the sense of a vector space not as a ordered set)
		\item We mean \ul{affinely independent} here, where the field elements (here $\R$) satisfy 
		\begin{align*}
			\sum s_i = 1
		\end{align*}
		this all ccould come from barycentric coordinates?!
		\end{itemize}
	}
\end{*remark}
So lets prove this little remark.
\begin{proof} % compare https://proofwiki.org/wiki/Intersection_of_Convex_Sets_is_Convex_Set_(Vector_Spaces)
	\scyllanote{Let $x,y \in \bigcap \CC$. Then we can use the definition of set intersection and get $\forall C \in \CC \colon x,y \in C$. Since all $C$ are convex we have
	\begin{align*}
		\for t \in [0,1] \colon (t-y)x + ty \in C.
	\end{align*}
	Therefore follows that also the elements in $\bigcap \CC$ are convex by set intersection. Hence $\bigcap \CC$ is convex.}
\end{proof}
\begin{proposition}[``connects'' linearly independent and affinely independent]
	Let $\set{x_0, \dots, x_p} \subseteq \Rn$. Then are equivalent:
	\begin{propenum}
		\item $x_1 - x_0, \dots, x_p - x_0$ are linearly independent,
		\item if $\sum s_i x_i = \sum t_i x_i$ and $s_i = t_i$, then $s_i = t_i$ for $i \in \set{0,\dots,p}$
	\end{propenum}
\end{proposition}
\begin{proof}
	Okay we have the set of points $\set{x_0, \dots, x_p}\subseteq \Rn$.
	\begin{itemize}
		\item $2. \implies 1.$: We know that $\sum t_i x_i = \sum s_i x_i$, so $\sum s_i = \sum t_i$ implies $s_i = t_i$ and this holds for all $i \in [p]$. So we get that 
		\begin{align*}
			\sum (s_i - t_i) x_i = 0 \with i \in [p]
		\end{align*}
		This is the definition of linearly independent. If i take an element in a linearly independent set away, the set is still linearly independent? Okayy there is something wrong still? What was the definition of lin. indep? again?!
	\end{itemize}
\end{proof}