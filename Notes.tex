% !TeX spellcheck = en_US
\documentclass[ngerman,a4paper,order=firstname]{mathscript}
\usepackage{mathoperators}
\usepackage{todonotes}

% % % local commands
\DeclareMathOperator{\Ad}{Ad}				% Adjoint
\DeclareMathOperator{\PSL}{PSL} 			% projective linear group 
\newcommand{\with}{\text{ with }}
\newcommand{\nd}{\text{ and }}
\newcommand{\for}{\text{ for }}
\renewcommand{\rhd}{\triangleright}
\renewcommand{\lhd}{\triangleleft} 			% normal subgroups
\DeclareMathOperator{\Set}{Set}				% Category of sets
\newcommand{\cat}{\mathcal C}				% some category
\DeclareMathOperator{\Vect}{Vect}			% Category of vector spaces
\DeclareMathOperator{\Grp}{Grp}				% Category of groups
\DeclareMathOperator{\QVect}{QVect}			% Cat of quasi vector bundles
\DeclareMathOperator{\Mod}{Mod}				% Cat of moduls
\newcommand{\EE}{\mathscr E}				% some cat E
\DeclareMathOperator{\Ann}{Ann}				% annihilator
\DeclareMathOperator{\morph}{Morph}				% cat of morphs
\newcommand{\Circlearrowleft}{\mathbin{\rotatebox[origin=c]{180}{$\circlearrowright$}}}
\DeclareMathOperator{\Cl}{Cl}				% conjugation class of something.
\renewcommand{\phi}{\varphi}				% always varphi for phi
\DeclareMathOperator{\Iso}{Iso}				% set of isomorphisms
\DeclareMathOperator{\Lin}{\mathscr L}		% set of continuous linear maps from V to W
\newcommand{\hset}[3]{\mathrm{H}^{#1}(#2;#3)} % set of equivalent cocycles over X
\newcommand{\isoset}[3]{\Phi^{#1}_{#2}(#3)} % isomorphism class of - F vector bundles
\DeclareMathOperator{\Open}{Open}				% inclusion category for presheafs
\newcommand{\presheaf}{\mathcal S}			% presheaf
\newcommand{\CC}{\mathcal C}

% % % % color note stuff
%\newcommand{\marganote}[1]{\textcolor{gray}{#1}}
%\newcommand{\fehmnote}[1]{\textcolor{red}{#1}}
\newcommand{\scyllanote}[1]{\textcolor{blue}{#1}}

% Nice looking emptyset
\let\oldemptyset\emptyset
\let\emptyset\varnothing

% get this stupid arrows:
%\usepackage{mathabx,graphicx}  % ---> add to mathoperators
%\def\Circlearrowleft{\ensuremath{%
%		\rotatebox[origin=c]{180}{$\circlearrowleft$}}}
%\def\Circlearrowright{\ensuremath{%
%		\rotatebox[origin=c]{180}{$\circlearrowright$}}}
%\def\CircleArrowleft{\ensuremath{%
%		\reflectbox{\rotatebox[origin=c]{180}{$\circlearrowleft$}}}}
%\def\CircleArrowright{\ensuremath{%
%		\reflectbox{\rotatebox[origin=c]{180}{$\circlearrowright$}}}}
%\begin{document}
%	\Huge
%	$\circlearrowleft \circlearrowright $
%	
%	$\Circlearrowleft \Circlearrowright $
%	
%	$\CircleArrowleft \CircleArrowright $

% % % local packages
\usepackage{braids}

\newlist{remarkenum}{enumerate}{1}
\setlist[remarkenum]{label=(\alph*),ref=\theremark~(\alph*)}
\crefalias{remarkenumi}{remark}

\newlist{propenum}{enumerate}{1}
\setlist[propenum]{label=(\alph*),ref=\theproposition~(\alph*)}
\crefalias{propenumi}{proposition}

\newlist{expenum}{enumerate}{1}
\setlist[expenum]{label=(\alph*),ref=\theexample~(\alph*)}
\crefalias{expenumi}{example}

\newlist{lemmaenum}{enumerate}{1}
\setlist[lemmaenum]{label=(\alph*),ref=\thelemma~(\alph*)}
\crefalias{lemmaenumi}{lemma}

\newlist{defenum}{enumerate}{1}
\setlist[defenum]{label=(\roman*),ref=\thedefinition~(\roman*)}
\crefalias{defenumi}{definition}

\newlist{theoenum}{enumerate}{1}
\setlist[theoenum]{label=(\roman*),ref=\thedefinition~(\roman*)}
\crefalias{theoenumi}{theorem}

\title{\textbf{Notes: Homology Theory by James Vick}}
\author{ScyllaHide}

\begin{document}
\pagenumbering{roman}
\pagestyle{plain}

\maketitle

\hypertarget{tocpage}{}
\tableofcontents
\bookmark[dest=tocpage,level=1]{Inhaltsverzeichnis}

\pagebreak
\pagenumbering{arabic}
\pagestyle{fancy}

\chapter*{Preface}
The plan is to go rather fast through the first chapter of this book(to get fast to K-Theory), take some notes, write down ideas, examples and remarks, rewrite proofs, so that i can understand them in my way. Also add sometimes reminders to concepts/definitions, so that i have a good overview about vector bundles and of course K-Theory. I will also use notation from courses i took in the past. But I will put remarks for the reader.
Hope you will find these notes helpful in any way.

ScyllaHide, \today
\chapter{Singular Homology Theory}
% !TeX spellcheck = en_US
\section{Singular Homology Theory}
Here we will introduce the singular homology theory of an arbitrary topological space. The ultimate goal is the \person{Mayer}-\person{Vietoris} sequence and a few applications like \person{Brouwer}-Theorem and nonretractibility of a disk onto its boundary.
So lets start with a few definitions
\begin{definition}
	We are in $\Rn$ and $x,y \Rn$ are points. Lets define a few things
	\begin{enumerate}
		\item \begriff{segment}
		\item \begriff{convex}
	\end{enumerate}
\end{definition}
\todo[inline]{Okay thats worth a proof!}
\begin{*remark}
	Thats crucial here, but need to show that $\bigcap$ of all convex sets in $\Rn$ which contain $A$ are again convex.
\end{*remark}

\part*{Anhang}
\addcontentsline{toc}{part}{Anhang}
\appendix

\nocite{*}
%\bibliography{literatur}
%\bibliographystyle{acm}

%\printglossary[type=\acronymtype]

\printindex

\end{document}